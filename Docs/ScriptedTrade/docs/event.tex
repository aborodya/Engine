An event is defined either as a scalar

\begin{minted}[fontsize=\footnotesize]{xml}
    <Event>
      <Name>Expiry</Name>
      <Value>2020-02-09</Value>
    </Event>
\end{minted}

or as an array of dates using ORE's ScheduleData node, e.g.

\begin{minted}[fontsize=\footnotesize]{xml}
    <Event>
      <Name>ExerciseDates</Name>
      <ScheduleData>
        <Rules>
         <StartDate>2016-02-06</StartDate>
         <EndDate>2016-05-06</EndDate>
         <Tenor>1D</Tenor>
         <Calendar>TARGET,US</Calendar>
         <Convention>F</Convention>
         <Rule>Forward</Rule>
       </Rules>
      </ScheduleData>
    </Event>
\end{minted}

using a rule based schedule or

\begin{minted}[fontsize=\footnotesize]{xml}
    <Event>
      <Name>ValuationDates</Name>
      <ScheduleData>
        <Dates>
          <Dates>
            <Date>2018-03-10</Date>
            <Date>2019-03-10</Date>
            <Date>2020-03-10</Date>
            <Date>2021-03-10</Date>
            <Date>2022-03-10</Date>
            <Date>2023-03-10</Date>
            <Date>2024-03-11</Date>
          </Dates>
        </Dates>
      </ScheduleData>
    </Event>
\end{minted}

using a list of dates. An array of dates can also be deduced from another, previously defined array by specifying a
shift rule which is useful e.g. to generate fixing or payment schedules from accrual schedules, or notification and
settlement schedules from exercise date schedules. Example:

\begin{minted}[fontsize=\footnotesize]{xml}
    <Event>
      <Name>FixingDates</Name>
      <DerivedSchedule>
        <BaseSchedule>AccrualDates</BaseSchedule>
        <Shift>-2D</Shift>
        <Calendar>TARGET</Calendar>
        <Convention>MF</Convention>
      </DerivedSchedule>
    </Event>
\end{minted}
