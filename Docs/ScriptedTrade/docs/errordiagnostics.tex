\subsection{Errors during parsing}

The log file contains errors that occur during parsing showing the code context and type of error, e.g.

\begin{minted}[fontsize=\footnotesize]{text}
DEBUG  Getting script 'EuropeanOption' from library
DEBUG  parsing script (size 313)
ALERT  an error occured during script parsing:
NOTICE parsing stopped at L1:1:0
NOTICE expected ")" in L4:68:0:
NOTICE       Option = LongShort * Quantity * PAY(( max( Payoff, 0 ), Expiry, Settlement, PayCcy);
NOTICE                                                             ^--- here
ALERT  [STEM] scripted trade could not be built due to parser errors, see log for more details.
\end{minted}

\subsection{Errors during runtime}

The log file also contains errors that occur during the script execution, e.g.

\begin{minted}[fontsize=\footnotesize]{text}
<<<<<<<<<<
             Payoff = PutCall * (Underlying(Expiry) - Strike);
             ======
>>>>>>>>>>
Error during script execution: variable 'Payoff' is not defined. at L3:14:6
\end{minted}

\subsection{Tracing}

If tracing is enabled in the engine parameters, the AST resulting from parsing is dumped to the log file, e.g.

\begin{minted}[fontsize=\footnotesize]{text}
 Sequence at L2:14:289
   DeclarationNumber at L2:14:35
     Variable(Payoff) at L2:21:6
       -
     Variable(ExerciseProbability) at L2:29:19
       -
   Assignment at L3:14:49
     Variable(Payoff) at L3:14:6
       -
     OperatorMultiply at L3:23:39
       Variable(PutCall) at L3:23:7
         -
       OperatorMinus at L3:33:29
         VarEvaluation at L3:34:18
           Variable(Underlying) at L3:34:10
             -
           Variable(Expiry) at L3:45:6
             -
         Variable(Strike) at L3:55:6
           -
   Assignment at L4:14:83
     Variable(Option) at L4:14:6
       -
     OperatorMultiply at L4:23:73
       Variable(LongShort) at L4:23:9
         -
       OperatorMultiply at L4:35:61
         Variable(Quantity) at L4:35:8
           -
         FunctionPay at L4:46:50
           FunctionMax at L4:51:16
             Variable(Payoff) at L4:56:6
               -
             ...
\end{minted}

This can be used to track down parsing errors at a low level. Furthermore the context before the script engine is run is
logged containing the variables from the data node of the scripted trade, e.g.

\begin{minted}[fontsize=\footnotesize]{text}
run script engine, context before run is:
Expiry                        (Event   )    const    February 9th, 2020
LongShort                     (Number  )    const    1.000000 (10000 det)
Option                        (Number  )             0.000000 (10000 det)
PayCcy                        (Currency)    const    USD
PutCall                       (Number  )    const    1.000000 (10000 det)
Quantity                      (Number  )    const    1000.000000 (10000 det)
Settlement                    (Event   )    const    February 9th, 2020
Strike                        (Number  )    const    2147.560000 (10000 det)
TODAY                         (Event   )    const    February 5th, 2016
Underlying                    (Index   )    const    EQ-SPX
\end{minted}

which can be used to verify the input data used for the concrete trade pricing. Finally each execution step is logged
with the AST node type currently processed, the value of the current evaluation and the code context which produced it

\begin{minted}[fontsize=\footnotesize]{text}
...
variable( Expiry ) at L3:45:6
expr value = February 9th, 2020
<<<<<<<<<<
             Payoff = PutCall * (Underlying(Expiry) - Strike);
                                            ======
>>>>>>>>>>

indexEval( EQ-SPX, February 9th, 2020 ) at L3:34:18
expr value = [1807.87,2534.44,1289.6,1541.2,3216.55,2120.7,1016.12,1159.3,2309.33,3898.24...]
<<<<<<<<<<
             Payoff = PutCall * (Underlying(Expiry) - Strike);
                                 ==================
>>>>>>>>>>
...
\end{minted}

If the Interactive flag is set in the engine parameters, the single execution steps are displayed on the console output
and the user is prompted for a command input which can be

\begin{itemize}
\item (c) to display the current state of the variables (context)
\item (q) to continue the script execution without being promted any further
\item return to execute the next step in the script
\end{itemize}
