In addition to the trade xml described in \ref{generalStructure},
\ifdefined\STModuleDoc
\ref{scriptNode}, \ref{dataNode}
\fi
we support an alternative, more compact, format where the variable names are derived from the node names and the type is given by an
attribute. The script
\ifdefined\STModuleDoc
must sit in the script library in this case (i.e. inlining is not possible) and
\fi
is referenced via a name derived from the root node of the trade data. Consider the following example of a one touch option in the
original format:

\begin{minted}[fontsize=\footnotesize]{xml}
<Trade id="SCRIPTED_FX_ONE-TOUCH_OPTION">
  <TradeType>ScriptedTrade</TradeType>
  <Envelope>
    <CounterParty>CPTY_A</CounterParty>
    <NettingSetId>CPTY_A</NettingSetId>
    <AdditionalFields/>
  </Envelope>
  <ScriptedTradeData>
    <ScriptName>OneTouchOption</ScriptName>
    <Data>
      <Event>
        <Name>Settlement</Name>
        <Value>2020-08-01</Value>
      </Event>
      <Event>
        <Name>ObservationDates</Name>
        <ScheduleData>
          <Rules>
            <StartDate>2018-12-28</StartDate>
            <EndDate>2020-08-01</EndDate>
            <Tenor>1D</Tenor>
            <Calendar>US</Calendar>
            <Convention>U</Convention>
            <TermConvention>U</TermConvention>
            <Rule>Forward</Rule>
          </Rules>
        </ScheduleData>
        <ApplyCoarsening>true</ApplyCoarsening>
      </Event>
      <Number>
        <Name>BarrierLevel</Name>
        <Value>0.009</Value>
      </Number>
      <Number>
        <Name>Type</Name>
        <Value>-1</Value>
      </Number>
      <Number>
        <Name>LongShort</Name>
        <Value>1</Value>
      </Number>
      <Number>
        <Name>Amount</Name>
        <Value>10000000</Value>
      </Number>
      <Currency>
        <Name>PayCcy</Name>
        <Value>USD</Value>
      </Currency>
      <Index>
        <Name>Underlying</Name>
        <Value>FX-TR20H-USD-JPY</Value>
      </Index>
    </Data>
  </ScriptedTradeData>
</Trade>
\end{minted}

In the compact format the same trade looks like this:

\begin{minted}[fontsize=\footnotesize]{xml}
<Trade id="SCRIPTED_FX_ONE-TOUCH_OPTION">
  <TradeType>ScriptedTrade</TradeType>
  <Envelope>
    <CounterParty>CPTY_A</CounterParty>
    <NettingSetId>CPTY_A</NettingSetId>
    <AdditionalFields/>
  </Envelope>
  <OneTouchOptionData>
    <Settlement type="event">2020-08-01</Settlement>
    <ObservationDates type="event">
      <ScheduleData>
        <Rules>
          <StartDate>2018-12-28</StartDate>
          <EndDate>2020-08-01</EndDate>
          <Tenor>1D</Tenor>
          <Calendar>US</Calendar>
          <Convention>U</Convention>
          <TermConvention>U</TermConvention>
          <Rule>Forward</Rule>
        </Rules>
      </ScheduleData>
      <ApplyCoarsening>true</ApplyCoarsening>
    </ObservationDates>
    <BarrierLevel type="number">0.009</BarrierLevel>
    <BarrierType type="barrierType">DownIn</BarrierType>
    <LongShort type="longShort">Long</LongShort>
    <Amount type="number">10000000</Amount>
    <PayCcy type="currency">USD</PayCcy>
    <Underlying type="index">FX-TR20H-USD-JPY</Underlying>
  </OneTouchOptionData>
</Trade>
\end{minted}

The supported types that must be specified in the \verb+type+ attribute are \verb+number+, \verb+event+,
\verb+currency+, \verb+dayCounter+ and \verb+index+. In addition we support some custom types that are mapped to numbers
internally and allow for a more natural representation of the trade:

\begin{itemize}
\item \verb+bool+ with a mapping \verb+true+ $\mapsto$ 1, \verb+false+ $\mapsto$ -1 
\item \verb+optionType+ with a mapping \verb+Call+ $\mapsto$ 1, \verb+Put+ $\mapsto$ -1, \verb+Cap+ $\mapsto$ 1,
  \verb+Floor+ $\mapsto$ -1
\item \verb+longShort+ with a mapping \verb+Long+ $\mapsto$ 1, \verb+Short+ $\mapsto$ -1
\item \verb+barrierType+ with a mapping \verb+DownIn+ $\mapsto$ 1, \verb+UpIn+ $\mapsto$ 2, \verb+DownOut+ $\mapsto$ 3,
  \verb+UpOut+ $\mapsto$ 4
\end{itemize}

Arrays of events are specified as in the example above (ObservationDates), for the other types the values are listed in
value tags, e.g. an array of numbers is declared as

\begin{minted}[fontsize=\footnotesize]{xml}
  <MyNumberArray type="number">
    <Value>100.0</Value>
    <Value>200.0</Value>
    <Value>200.0</Value>  
  </MyNumberArray>
\end{minted}
