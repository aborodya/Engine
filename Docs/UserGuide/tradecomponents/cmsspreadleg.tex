\subsubsection{CMS Spread Leg Data}
\label{ss:cmsspreadlegdata}

Listing \ref{lst:cmsspreadlegdata} shows an example for a leg of type CMSSpread.

\begin{listing}[H]
%\hrule\medskip
\begin{minted}[fontsize=\footnotesize]{xml}
      <LegData>
        <LegType>CMSSpread</LegType>
        <Payer>false</Payer>
        <Currency>GBP</Currency>
        <Notionals>
          <Notional>10000000</Notional>
        </Notionals>
        <DayCounter>ACT/ACT</DayCounter>
        <PaymentConvention>Following</PaymentConvention>
        <ScheduleData>
          ...
        </ScheduleData>
        <CMSSpreadLegData>
          <Index1>EUR-CMS-10Y</Index1>
          <Index2>EUR-CMS-2Y</Index2>
          <Spreads>
            <Spread>0.0010</Spread>
          </Spreads>
          <Gearings>
            <Gearing>8.0</Gearing>
          </Gearings>
          <Caps>
            <Cap>0.05</Cap>
          </Caps>
          <Floors>
            <Floor>0.01</Floor>
          </Floors>
          <NakedOption>false</NakedOption>  
        </CMSSpreadLegData>
      </LegData>
\end{minted}
\caption{CMS Spread leg data}
\label{lst:cmsspreadlegdata}
\end{listing}

The elements of the CMSSpreadLegData block are identical to those of the CMSLegData (see \ref{ss:cmslegdata}), except
for the index which is defined by two CMS indices as the difference between \verb+Index1+ and \verb+Index2+.

